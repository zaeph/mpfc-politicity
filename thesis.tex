\documentclass[
,a4paper
,DIV=12
,12pt
,abstract
,bibliography=totoc
]{scrartcl}

\usepackage[
,font=guyot
,babel=french
,header=false
,geometry
,autolang=hyphen
,numbers=osf
]{sty/zpart}



\author{Leo Vivier}
\date{\today}
% \date{\printdate{2018-05-15}}
\title{Monty Python's Flying Circus}
\subtitle{Political Satire \& Political Posture}

\addbibresource{/home/zaeph/org/bib/monty-python.bib}
\addbibresource{/home/zaeph/org/bib/monty-python-didactics.bib}
% \nocite{*}

% \AtBeginEnvironment{quote}{\vspace{-0.2\baselineskip}\singlespacing}
\AtBeginEnvironment{quote}{\vspace{-0.2\baselineskip}}
\AfterEndEnvironment{quote}{\vspace{-0.2\baselineskip}}
\AtBeginEnvironment{quotation}{\addvspace{\baselineskip}\singlespacing}
\AfterEndEnvironment{quotation}{\addvspace{1.5\baselineskip}}

% Redefine parts
\renewcommand\partheadstartvskip{\clearpage\null\vfil}
\renewcommand\partheadmidvskip{\par\nobreak\vskip 20pt}
% \renewcommand\partheadmidvskip{\par\nobreak\vskip 20pt\thispagestyle{empty}}
\renewcommand\partheadendvskip{\vfil\clearpage}
\renewcommand\raggedpart{\centering}

% Reduce quote font-size
\AtBeginEnvironment{quote}{\vspace{0.3\baselineskip}\setstretch{1.2}\small}

% \linespread{1.7}

\begin{document}
% \onehalfspacing
% \doublespacing

\maketitle

\begin{abstract}
  Cet article explore le contenu politique de la série \emph{Monty Python’s Flying Circus} afin de déterminer si elle est le support d’une critique politique, et donc d’une intention manifeste des auteurs de \emph{faire du politique}.  La tension principale de l’étude réside dans le dévoiement du terme \enquote{apolicité} utilisé dans le manifeste \emph{Pythonesque} pour exprimer une distanciation avec les partis institutionnels plutôt qu’une neutralité politique.  La conclusion est que le \emph{Flying Circus} est bel et bien le théâtre d’une critique politique égalitaire et antiautoritaire, sous couvert d’autotélie.
\end{abstract}

% \begin{otherlanguage}{english}
%   \begin{abstract}
%   \end{abstract}
% \end{otherlanguage}


\tableofcontents
\clearpage

\setstretch{1.8}

\selectlanguage{english}
\section*{Introduction}
\addcontentsline{toc}{section}{Introduction}
\label{sec:proposal}

\emph{Monty Python’s Flying Circus} holds a particular place within the Britcom\footnote{Although the term \enquote{\emph{Britcom}} is sometimes read as a portmanteau of ‘British’ and ‘sitcom’, it is used here in its broadest meaning of ‘British comedy’.} canon.  Broadcast on the BBC from 1969 to 1974 at a time where the channel was hoping to stave off the commercial competition, the series pioneered a genre of British humour whose novelty only equalled its perplexity.  The Pythonesque---as the genre came to be called---gathered little traction in its earliest iterations on the small screens, but it gained considerable prominence with the release in 1975 of their second film, \emph{Monty Pythons and the Holy Grail}.  Since then, the genre has grown organically, having a lasting effect on other genres and media through which it permeated.  So sprawling was its influence that the Pythonesque has grown beyond the easily definable, and any attempt to circumscribe it often devolves into a haphazard enumeration of features.  This plurality finds parts of its justification in the interactions between the different writing teams\footnote{The two main writing teams were the ‘Cambridge’ team (John Cleese \& Graham Chapman) and the ‘Oxford’ team (Terry Jones \& Michael Palin), based on the universities which the Pythons attended.  Eric Idle and Terry Gilliam mostly worked on their own.} within Monty Python, resulting in the \emph{Gestalt} that the Pythonesque has now become.  However, by way of conflation, the label fails to account for some of the finer discrepancies between the members, and this is particularly tangible in the way the teams approached political satire.

The term \emph{political satire} warrants clarification.  Compositionally, one might infer that a \emph{political} satire is a satire that \emph{is} political, and whilst it may be true at times, by no means is it always the case.  Instead, a political satire should be construed as a satire which \emph{derives} its entertainment from politics.  In other words, political \emph{satire} does not necessarily imply political \emph{protest}, thereby allowing for, in a puzzling way, an apolitical political satire (i.e., by putting it all together, a satire deriving its entertainment from politics without demonstrating a political stance).  This semantic gymnastics might seem superfluous, but it is key to understanding how the Pythons denied the political content of their artistic endeavours, particularly evident in the following address by Eric Idle:

\begin{quote}
  \label{quote-anti-authoritarian}
  Comedy’s job is to be against things, not for them.  \emph{Monty Python was firmly apolitical, though anti-authoritarian in flavour}.  In the years in which it flourished it was no longer possible to take any party seriously.  Thus, \emph{the Python attack is fixed on all authority figures involved in growing up in this country}: teachers, policemen, judges, mothers, minor royalties, politicians, army officers, even those in charge of the BBC; and it consequently aroused the anger of the \emph{middle classes}\footnote{\Cite[42]{clifford1989}.}. [emphasis added]
\end{quote}

The tension highlighted by Idle between the \enquote{apolitical} and the \enquote{anti-authoritarian} is not evident.  Since anti-authoritarianism \emph{is} a political stance, i.e., one’s positioning against authoritarian political practices, \enquote{apolitical} has to be understood in its other acceptation, namely a disenchantment with party politics\label{disenchantment}.  This is overtly mentioned in \enquote{it was no longer possible to take any party seriously}, but also, to a lesser degree, in \enquote{[c]omedy’s job is to be against things, not for them}.

Idle is quick to buttress his argument by invoking a class conception of society, which further establishes the political aspect of his comment.  He links the denouncement of \enquote{authority figures} to \enquote{the anger of the middle classes}, thereby implicating a sense of complacency from the latter towards the former.  In one of the \emph{Flying Circus} sketches\footnote{\emph{Communist Quiz}, AKA. \enquote{World Forum}.  In: \emph{Flying Circus}, Episode 25.  First broadcast on 15 December 1970.} featuring the very same Eric Idle as a game show host quizzing Karl Marx (Terry Jones) for a chance to win a \enquote{red lounge suite}, the following dialogue ensues:

\begin{quote}
  \textbf{Presenter}: […] You're on your way to the lounge suite, Karl. Question number two. The struggle of class against class is a \emph{what} struggle? A \emph{what} struggle?\\
  \textbf{Karl}: A \emph{political} struggle.\\
  \emph{(Tumultuous applause.)}\\
  \textbf{Presenter}: Yes, yes! […]
\end{quote}
By mentioning \enquote{[t]he struggle of class against class}, the question clearly situates the propos within a Marxist framework, even if its \emph{raison d’être} is to be derided within the sketch.  The answer to the question firmly places the notion of class within the political realm.  Going back to Idle’s usage of the term \enquote{apolitical}, if it were to be taken as a lack of political stance, it would be hard to reconcile how their comedy would \enquote{arous[e] the anger of the middle classes}.  If a show predominantly raises the ire of a given social class, it would be fair to assume that this show demonstrates a political bias, even if it is denied by the people making it.  However, there is a thin line between political \emph{bias} and political \emph{affiliation}, especially when it comes to the reception of a politically-charged art piece.  A significant path for further study would be to approach this cognitive dissonance and to explore how its evolution coloured the productions of the comedy group.

Aside from giving an \enquote{anti-authoritarian flavour} to their \enquote{apolitical} satire, the Pythons also had a desire to subvert the genre altogether.  \textcite{geng1990} provides a description of their stance:

\begin{quote}
  \label{quote-idle-autotelic}
  The Pythons were trying to \emph{resist} what is usually meant by satire.  […] Monty Python was more interested in a truth that satirists hate to think about: \emph{people don’t want to change their minds and rarely change them in response to the lessons of satire}.  It’s hard to face this without getting cynical.  Positively embracing it is the heart of the Pythons’ style. [emphasis added]
\end{quote}
Conflating those two quotes, the Python project becomes a little clearer: to subvert as many conventions of the medium (television) and the genre (comedy) as they can get away with, and to not have a political agenda.  In other words, the project is presented as being strictly autotelic, i.e., of existing for its own artistic sake rather than for serving a purpose, political or otherwise.  This view was held firmly by every member of the group, and this constant desire for subversion and innovation is what ultimately led to the end of Monty Python’s experience on the small screens.  Whilst \emph{Series 1} to \emph{3} have 12 episodes each, \emph{Series 4} was limited to six due to, on top of John Cleese’s departure from the show, the increasing difficulty they faced trying to meet the standards they had set for themselves.  The group went on to produce films which have now cemented their place within the collective imaginary, but almost everything that works within those longer productions was first experimented upon within their series.  Therefore, the \emph{Flying Circus} enables an archaeological approach to Monty Python’s success.

Going back to the political satire, the description of their humour as being \enquote{apolitical} might come as a surprise to many, especially since a number of their famous sketches are seen as being directly political.  For example, the film \emph{Monty Python and the Holy Grail} (1975) features a scene where Arthur (Graham Chapman) confronts a group of anarcho-syndicalist peasants scavenging through the mud (Michael Palin and Terry Jones), one of which (Palin) gives him a lecture on social class:

\begin{quote}
  \textbf{Dennis}: What I object to is that you automatically treat me like an inferior!

  \textbf{Arthur}: Well, I am King!

  \textbf{Dennis}: Oh, King, eh, very nice. And how d'you get that, eh? \emph{By exploiting the workers! By 'anging on to outdated imperialist dogma which perpetuates the economic and social differences in our society. If there's ever going to be any progress with the}---

  \textbf{Woman}: Dennis, there's some lovely filth down here. \emph{(Upon noticing Arthur.)} Oh! How d'you do?

  [emphasis added]
\end{quote}
The clear disconnection between the subject matter and the environment within which it takes place is quite evident: a class-conscious peasant of the Middle-Ages harping on about \enquote{exploiting the workers} and \enquote{imperialist dogma} whilst facing the embodiment of the Enemy makes for quite a memorable scene.  However, one could argue that the memorability of the scene is not as much assured by the disconnection between subject matter and environment as it is by the nature of Dennis’s arguments.  The words used by the character clearly situates his propos within the intellectual counter-culture of the 1960s in Britain, more specifically on the creation of communes.  By 1975, the movement was by and large past his prime, but the novel ideas which it introduced have had an enduring stay within the imaginary of the non-institutional Left.  It is hard to put into words how striking it is to see those ideas approached within a popular\footnote{Referring to \emph{Monty Python and the Holy Grail} as a \emph{popular} film might be a stretch, considering its humble beginnings as a cult film.  However, it would be hard to argue that the film has not entered the collective imaginary as \emph{the} prototypical example of British comedy.} film.  It is a testament to how different the political landscape of the late 1960s was compared to today.  Even if the Pythons intended this depiction to be humorous, mocking the way a British hippie would have presented their anti-imperialistic views, the sketch has immortalised those positions into a format which is far more respectful of the political ideas of the Left than the modern critical representations of the hippie have since become, most of them centring their critique on the consumption of drugs.

A point can therefore be made on how the Pythons engaged with the ideas of their time, which does not necessarily contradict their apolitical avowal.  By deciding to platform those ideas in a way that allows them to be entertained by the viewer, even if it is done under the guise of a jab, it is the position of the author that the Pythons had a certain sympathy for the Left which coloured its representation throughout their works.  It bears repeating that the members had different political positions of their own, and that that some of those evolved quite significantly in the wake of the 1970s and of the Thatcher years.  However, it is fair to assume that they were all exposed to the ideas of the Left during their years at university, most likely through the flourishing counter-cultures of the 1960s.  So, although they all originated from the middle classes, the wind of change that blew on Britain during that period allowed for some of those novel ideas to take roots within the political and societal representations of the members.

The counter-cultures and their inherent opposition to the norm and the Establishment can also be understood as a fundamental influence for the Pythons’s attitudes towards subversion.  Whilst their politically-inclined peers at university endeavoured to usher in the new world by subverting the practices and the expectations of their time, the Pythons deterritorialised those attitudes towards the institutions to the realms of comedy and television.  Therefore, although the individual political sensibilities of the members have evolved throughout the decades, the Python project is grounded in a political attitude of anti-Establishment.

The purpose of this preliminary research will be to ascertain whether the political content of \emph{Monty Python’s Flying Circus} can be described as satire despite the autotelic and apolitical intentions of the authors.  To do so, a theoretical framework is devised by conflating two sources: \textcite{declercq2018}, and \textcite{mcgowan2017}.  The first section presents the definition of satire that is proposed by \textcite{declercq2018}, and the second section explores the links between satire and comedy as discussed in \textcite{mcgowan2017}.  The conclusion prepares further research.

\section{\textcite{declercq2018}: Framing Satire}
Defining \emph{satire} is not an easy matter.  \textcite{elliott1962} provides the following justification:

\begin{quote}
  \label{quote-satire-elusive}
  No strict definition can encompass the complexity of a word that signifies, on one hand, a kind of literature […] and, on the other, a mocking spirit or tone that manifests itself in many literary genres but can also enter into almost any kind of human communication.
\end{quote}
An attempt was made in the introduction to frame the Pythonesque, but \emph{satire} itself is just as elusive.  This observation is the starting point of \textcite[319]{declercq2018}: \enquote{[t]here has been a growing scholarly consensus that a definition of satire, which identifies necessary and sufficient conditions, is impossible}.  The solution has often been to \enquote{settle for a characterization of satire through a family-resemblance cluster of nonessential features}\footnote{\Cite[319]{declercq2018}.}, just as it has been with the Pythonesque and most other preliminary studies in comedy.  In spite of this premiss, the article sets out to provide a definition of satire which \enquote{identifies the \emph{central dynamic} in satire between its moral function as \emph{critique} and aesthetic function as \emph{entertainment} [emphasis added]}\footnote{\Cite[319]{declercq2018}.}.  Besides considering \emph{critique} and \emph{entertainment} as the core ingredients of satire, \textcite{declercq2018} presents two versions of the proposal: a \emph{weak} version which assumes that intended critique and entertainment are \emph{necessary conditions} for satire, and a \emph{strong} version which posits that \enquote{these conditions are also \emph{jointly sufficient} for satire [emphasis added]}\footnote{\Cite[319]{declercq2018}}.  Whilst the approach might seem rigorous because the definition is presented as a formal rule with conditions, it offers a surprising amount of flexibility.

In the introduction, \emph{political} satire was presented as \enquote{a satire which \emph{derives} its entertainment from politics}, which covers the \enquote{entertainment} part of \textcite{declercq2018}’s proposal.  However, it is important not to conflate \emph{entertainment} with \emph{comedy} so as to not limit the potential of the latter term\footnote{\label{note-satire-vs-comedy}Moreover, \textcite[320]{declercq2018} also questions the centrality of humour within the definition of satire, which leads to a problematisation of the links between satire and comedy.  Orwell’s \emph{1984} is provided as an example of a satire that is not a comedy.}.  Also, whilst the term \enquote{\emph{political} satire} was used to clarify the targeted object of the satire, it should not imply the existence of non-political satires, and the desire to infirm this hypothesis might stem from a misunderstanding of the terms.  \enquote{Non-political} is as ambiguous as \enquote{apolitical}, since the acceptations of both terms are heavily influenced by what \enquote{political} or \enquote{\emph{the} political} is perceived to mean in one’s idiolect.  It is outside the scope of this research to explore the impact that such beliefs have on the reception of satire as well as comedy---although the topic is sure to be constructive---and a derived version of the radical truism \enquote{everything is political} will have to be supposed.  More specifically, in light of what was discussed in the introduction, anti-authoritarianism would be construed as a political stance regardless of one’s sentiment towards institutional politics.  Going back to the misunderstandings that might stem from \enquote{political satire}, \enquote{satire} is an established literary genre dating back to Ancient Greece and Ancient Rome, and of which there were three common types: Horatian, Juvenalian, and Menippean.  Again, it is besides the scope of this research to present the different types of satire\footnote{A succinct description of the Latin types---Horatian and Juvenalian---can be found in \textcite[104--106]{stott2014}.  A framework for understanding Menippean satire is explored in \textcite{bakhtin1984}.  In it, Menippean satire is conflated with the Carnivalesque, which would be particularly germane within a literary study of the \emph{Flying Circus}.}, but it is important to note that the literature ascribes different levels of politicity\footnote{The term \enquote{politicity} has to be understood as \enquote{the political character of a medium}.  The term comes from \textcite[23]{ranciere2000}, but the term is not defined in the \emph{Oxford English Dictionary}, nor is \enquote{\emph{politicité}} defined in \emph{Le Grand Robert}.} to each type.  This last point could lead to a constructive discussion on the idea that modern televised satire has \enquote{lost its edge}\footnote{Sophia A. McClennen’s work would be of particular import for a comprehensive study of the topic, most notably \textcite{mcclennen2013, mcclennen2014, brasset2019}.}, but more importantly, for the study at hand, it stresses the rôle of intention in the creation and reception of satire.

Going back to the central dynamic in satire between entertainment and critique to now focus on the latter, \textcite[321]{declercq2018} posits the following:

\begin{quote}
  [C]ertain types of comedy about politics that lack a moral dimension are often casually identified as satire, but really are something else. In this respect, [\textcite[23--26]{peterson2008}] has argued that \emph{Saturday Night Live} or Jay Leno’s monologues are really \enquote{\emph{pseudo-satire}} because they generally \emph{ridicule politics without taking a moral stand}.  [emphasis~added]
\end{quote}
To reuse the terms which were used in the formal definition, a \enquote{moral stand}---which is an intention---is presented as a necessary condition for critique\footnote{\textcite[323]{declercq2018} defines critique \enquote{as a committed moral opposition against a target, sustained by an analysis of that target’s perceived social wrongness}.} and, therefore, as a necessary condition for satire.  The use of \enquote{\emph{moral} stand} as opposed to \enquote{\emph{political} stand} is yet another example of the aforementioned ambiguity of the term, but \enquote{moral} has the benefit of being less equivocal than \enquote{political}.  To illustrate the casual identification of political comedy as satire, \textcite[321-322]{declercq2018} proceeds to compare the difference in treatments of a same topic---the 2014 Fifa World Cup in Brazil---by two self-identified satirical shows---\emph{Last Week Tonight}, and \emph{Mock the Week}.  The conclusion is that, \enquote{although both \emph{Mock the Week} and \emph{Last Week Tonight} serve a function as entertainment, the former lacks the latter’s additional function as critique}\footnote{\Cite[322]{declercq2018}.}.  In terms of reception, \textcite[322]{declercq2018} argues that \enquote{marketing \emph{Mock the Week} as satire introduces \emph{expectations of a critical purpose} which the show does not set out to fulfill and therefore makes it seem artistically worse than it is}, which does not prevent \emph{Mock the Week} and other shows like it from being very popular with certain fringes of the population, most notably the young and the Left-leaning\multfootnote{Both the institutional Left and the radical Left.;Whilst the age bias can probably be accounted for by the traditional intergenerational divide between a younger generation and the one that brought it up, the Left-leaning bias would be an exceedingly interesting subject to study within a European context.  Most of the literature \parencite[e.g.][]{brasset2019, mcclennen2014, kercher2006} seems to have been focussed on North-America which has a completely different political landscape, most notably with the absence of a true institutional Left within the bipartisan paradigm.  Still on the topic of the Left and humour, the emancipating rôle of comedy is explored in \textcite{mcgowan2017}, which is presented later in this study.}.  Furthermore, it could be argued that the recent commercial co-optation of the term \enquote{satire} is the primary reason why the mordant humour that formerly used to be veneer of satire eventually became its sole defining factor.  \textcite[322]{declercq2018} rightfully mentions that \enquote{[s]uch conflation with malicious shock humor contributes to confusion about the political significance of satire}, although the point could be expanded to include any type of political comedy, which is done in the next section of the current study.  To sum up, \textcite{declercq2018} provides the necessary framework to distinguish satire from \emph{pseudo}-satire, the difference being that the latter lacks a moral stand.

For the study at hand, it implies that \emph{Monty Python’s Flying Circus} could only be considered as a genuine satire if a moral stand can be discerned within their shows.  However, this is where things become complicated when one has to consider what constitutes a \enquote{\emph{moral} stand}.  Regardless of how nebulous the term might appear to be, within the central dynamic of satire, it could be essentialised to an \emph{intention} to \emph{criticise} an object---animate or not.  For the picture to be complete, one also has to consider the other pole of the dynamic which implies an intention to \emph{entertain}\footnote{The \enquote{intention to entertain} should not be conflated with the comedic project.  The distinction between satire and comedy in \textcite{declercq2018} is mentioned in note \ref{note-satire-vs-comedy}.}.  Focussing on the latter, this would not be very far from questioning the rôle of intention within the creation and the reception of the Arts, but far too many souls have already been agitated by the matter before and since Barthes\multfootnote{\Cite{barthes1968}.;On the topic of intractable problems, one could even be so bold as to question what constitutes entertainment within the reception of the Arts, which is exactly what \textcite[322--329]{declercq2018} sets out to do.  As eager as the author of the current study would be to pore over ideas such as \enquote{[e]ntertainment is an aesthetic classification} \parencite[323]{declercq2018} or of humour serving a \enquote{solemn and didactic aesthetic project that cultivates bemusement and desperation} \parencite[325]{declercq2018}, it is beyond the scope of the current research which is mostly concerned with the political pole of satire.}.

What is of more import for the current study is that \textcite[322--323]{declercq2018} posits that, although the \enquote{purposes [of critique and satire must] necessarily interact, […] neither is wholly instrumental to the other}\footnote{\Cite[323]{declercq2018}.}.  \textcite[325]{declercq2018} further specifies \enquote{that satire is not critique \emph{through} entertainment---even if satirists like Horace sometimes present it as such to legitimize their practice}, and adds that \enquote{a modern satirist like John Oliver [the host of \emph{Last Week Tonight}] stresses that he simply revels in \emph{\enquote{spectacle} for the sake of it} [emphasis added]}\footnote{\Cite{declercq2018,marchese2016}.}.   The key takeaway from those quotes is that an autotelic avowal might be a factor of legitimisation within the realm of political comedies.  This theory also has the benefit of having interesting diachronic ramifications with the previously-explored commercial co-optation of the term \enquote{satire}.  In other words, whilst pseudo-satirists seem to revel in the commercially-viable posturing of dissension, some genuine satirists seem to have, as a reaction, distanced themselves not only from the term \enquote{satire}, but also from clearly subscribing to any political endeavour\footnote{\enquote{Endeavour} should be understood here in its broadest sense, i.e., not only as a political campaign.}, let alone initiating them\footnote{This is a direct echo to Idle’s quote on the Pythons’s \enquote{trying to resist what is usually meant by satire} \parencite[quote fully-reproduced on p.~\pageref{quote-idle-autotelic}]{geng1990}.}.  This is not to say that those artists have ceased to be political, but rather that they have ceased to be \emph{overtly} so.  One might be tempted to rationalise both the masquerading as satire (for pseudo-satirists) and the camouflaging of its moral stand (for genuine satirists) as being motivated by \emph{consensuality} and the markets of views.  Whilst it might definitely be true for the former, this theory would fail to elucidate the complexity of the attitudes of modern satirists towards the political\footnote{The current study will not permit a full genealology of this aversion for the political which is not only shared by satirists, but by the population at large.  It is the belief of the author that this wrongfully-heralded \emph{depoliticisation of the masses} comes from the general disenchantment with party politics which was mentioned in the introduction (p.~\pageref{disenchantment}).  Furthermore, this disenchantment would itself be the result of the quasi-oligarchical devolution of most modern neoliberal states, and the distance that it has introduced between constituents and rulers.}.  However, an element of response might be found in the inherent politicity of comedy.

\section{\textcite{mcgowan2017}: Satire \& Comedy}

The goal that \textcite{mcgowan2017} sets out to accomplish is to provide a modern theory of \emph{comedy}.  Just as \textcite{declercq2018} did with \enquote{satire}, the study is not limited to a theory of the literary genre referred to by \enquote{comedy}, although it informs the theory where relevant.  Rather, \textcite{mcgowan2017} is concerned with comedy in its broadest sense which blurs the lines between \enquote{comedy}, \enquote{laughter}, and \enquote{humour}.  In that sense, it raises similar concerns as those voiced in \textcite{elliott1962} on the elusiveness of a \enquote{strict definition} of satire\footnote{The quote was fully-reproduced on p.~\pageref{quote-satire-elusive}.}.  Whilst \textcite{declercq2018} opted for a formal definition of satire which essentialised it as dynamic between critique and entertainment, the approach in \textcite{mcgowan2017} is more philosophical in nature, positing that \enquote{[o]ur widespread engagement with comedy bespeaks our desire to \emph{philosophize without actually becoming philosophers} [emphasis added]}\footnote{\Cite[179]{mcgowan2017}.}.  The author further specifies:

\begin{quote}
  Whereas philosophy requires some degree of learning, one need not even be literate to participate in the speculative enterprise of comedy. Though philosophy speculates more directly, comedy offers a more accessible version of speculative thought.\footnote{\Cite[179--180]{mcgowan2017}.}
\end{quote}
The notion of \enquote{accesib[ility]} is of particular import, and it complements the point made on the popularity of pseudo-satire in the previous section: for a medium to be \emph{popular}, it has to be perceived as \emph{accessible}.  This idea was conceptualised as early as Aristotle\footnote{\Cite{aristotle2004}.} which mapped \emph{phaulos} (\enquote{lowbrow} or \enquote{playful}) to comedy, and \emph{spoudaios} (\enquote{highbrow} or \enquote{serious}) to tragedy, thereby providing the perfect excuse for the \emph{literariat}\footnote{The word is made-up.  It should be understood as being to literature what the \enquote{commentariat} is to the news.  \enquote{Belletrists} is equally pejorative, but centres its critique on the \enquote{substance vs. style} axis, whereas \enquote{literariat} centres its critique on elitism and cultural supremacy.} to dismiss the latter for the longest time.  Whilst the rise of the cultural studies in the 1970s seems to have largely antiquated those attitudes, comedy remains to this day critically understudied.

For \textcite[179]{mcgowan2017}, \enquote{[t]he failure to take comedy seriously is the most damaging attitude to take toward it, and yet this attitude is almost ubiquitous}.  However, the finality of the argument is that the trivialisation of comedy might be the very reason why it proves to be so effective in changing mentalities: \enquote{experiencing comedy is always an \emph{existential act} that forces us to confront the basic structure of our \emph{subjectivity} [emphasis added]}\footnote{\Cite[179]{mcgowan2017}.}.  In other words, \enquote{[c]omedy is an encounter with the fundamental contradiction of our subjectivity}\footnote{\Cite[181]{mcgowan2017}.}.  This philosophical consideration is at the core of the theory, and whilst it informs all of its facets, it proves invaluable in defining the inherent politicity of comedy, which can be explained by the overlap between \emph{moral} philosophy and \emph{politics}\footnote{\enquote{\emph{Moral} philosophy} is defined as \enquote{[t]he knowledge or study of the \emph{principles} of human action or conduct [emphasis added]} \parencite{oed-philosophy}, and \enquote{politics} can be understood as the actualisation of those \enquote{principles} in society.}.  If every act of comedy can be construed as having philosophical undertones, surmising that it also has political overtones would not be far-fetched, and might even complete the harmony.  Furthermore, not only might it put into question the validity of the label \enquote{\emph{political} comedy}, but it might also herald a future commercial co-optation of the term, similarly to what happened to \enquote{satire}.

Going back to the facets of comedy explored in \textcite{mcgowan2017}, two are of particular interest on the topic of \emph{Monty Python’s Flying Circus}: \enquote{Lack \& Excess}\footnote{\Cite[19--48]{mcgowan2017}.}, and \enquote{Ideology \& Equality}\multfootnote{\Cite[161--177]{mcgowan2017}.;Every chapter in \textcite[19--48]{mcgowan2017} is presented as a tension between two terms, which makes for a particularly dynamic argument.}, but only the latter is relevant on the topic of the series’s politicity\footnote{\enquote{Lack \& Excess} would be particularly useful for a taxonomical approach to the comedy of the \emph{Flying Circus}.}.  Whilst the author posits that, \enquote{[e]ven in its most banal form, comedy is freedom from hierarchy}\footnote{\Cite[161]{mcgowan2017}.}, and that it \enquote{liberates us from the constraints that govern our everyday life}\footnote{\Cite[161]{mcgowan2017}.}, the chapter introduces a dichotomy between \enquote{egalitarian} and \enquote{ideological} comedies.  The difference between the two is that, even though they share the same kernel of \emph{subversion} that is common to all comedies, it serves a different purpose.  If \enquote{egalitarian} comedy is qualified as \enquote{class warfare with indirect means}\footnote{\Cite[162]{mcgowan2017}.}, which would agree with the idea of \emph{political} subversion, \enquote{ideological} comedy co-opts this subversion to reinforce the \emph{status quo}.  This dynamic seems reminiscent of the one which was explored between satire and pseudo-satire insofar as a similar political posturing was at stake in their discrimination.  The conflation of those two observations proves useful in situating where modern comedic satires stand in respect to the authorities, whether they be the channels that broadcast them or the State at large.  More specifically, it would contradict the idea formulated in \textcite[321]{declercq2018} that pseudo-satire lacks a \enquote{moral dimension}, surmising in its stead that not having a moral stand \emph{is} a moral stand\footnote{This is reminiscent of Sartre’s position on choice: \enquote{\foreignlanguage{french}{[l]e choix est possible dans un sens, mais ce qui n'est pas possible, c'est de ne pas choisir. Je peux toujours choisir, mais je dois savoir que si je ne choisis pas, je choisis encore}} \parencite{sartre2018}.}, namely one that favours consensuality and the \emph{status quo}.  Similarly, the egalitarian agenda that permeates both egalitarian comedy and satire alike \emph{is} a moral stand, regardless of how eager the authors might be to ascribe it to \enquote{common sense}, thereby somewhat depoliticising it\footnote{The appeals to \enquote{common sense} and rationality are especially damaging in the context of modern politics and its relation to truth and storytelling.  On regimes of truth, see \textcite{foucault1975, lorenzini2015}.  On political storytelling, see \textcite{salmon2019, salmon2008}.}.

\textcite[163]{mcgowan2017} argues that the difference between \enquote{egalitarian} and \enquote{ideological} comedies cannot be adequately explained by merely considering the relationship between the source and the target of the comedy.  Instead, the author argues that, \enquote{[s]ince seemingly egalitarian comedy can have an ideological effect, the evaluation of comedy must examine not only its source or object but take into account its \emph{effects} [emphasis added]}\footnote{\Cite[164]{mcgowan2017}.}.  Those \enquote{effects} are further specified in terms of \enquote{division} and \enquote{wholeness}:

\begin{quote}
  [C]omedy has the \emph{ability} to reveal \emph{division} or splitting where we perceive \emph{wholeness}.  When it sustains this revelation, it functions successfully as egalitarian comedy.  Egalitarian comedy exposes the contradictions of the social order and of the subject who exists within this order.  In egalitarian comedy, both the source of the comedy and its target appear divided internally.  It is the emergence of this internal division that enables us to laugh while also \emph{facilitating critique}\footnote{\Cite[164]{mcgowan2017}.}.  [emphasis added]
\end{quote}
If \enquote{the \emph{ability} to reveal division} is understood as \enquote{subversion}, a clear link is made between \emph{subversion} and \emph{critique}.  Furthermore, if comedy is both inherently subversive and inherently political, it might be possible to infer that comedic subversion is itself inherently political, blurring the lines between \emph{subversion} and \emph{transgression}.  It is this crucial point that excludes the possibility of autotelic subversion, a tenet of the Pythonesque according to its forerunners.

\section*{Conclusion}
\addcontentsline{toc}{section}{Conclusion}
By conflating both the findings in \textcite{declercq2018} and \textcite{mcgowan2017}, the idea that \enquote{Monty Python was firmly apolitical, though anti-authoritarian in flavour}\footnote{\Cite[42]{clifford1989}.  The quote was fully-reproduced on p.~\pageref{quote-anti-authoritarian}.} can now be revisited.  The demonstration should have made it clear that anti-authoritarianism is both political (\enquote{authoritarianism}) \emph{and} subversive (\enquote{anti-}), which not only invalidates apoliticity \emph{stricto sensu}, but also posits an opposition to the \emph{status quo} (\enquote{authorit[y]}), which is taken to mean institutional politics.

It is obvious that the claims made in this study ought to be substantiated by an analysis of the series itself.  An earlier draft had set out to analyse three representative sketches:

\begin{itemize}
\item \enquote{Tuesday Documentary / Children's Story / Political Broadcast}.\\
  In \emph{The War Against Pornography},  Episode 32 (Season 3, Episode 6), 13:45--17:10.
\item \enquote{Politicians - An Apology}.\\
  In \emph{The War Against Pornography},  Episode 32 (Season 3, Episode 6), 17:10--18:13.
\item \enquote{Party Political Broadcast (choreographed)}.\\
  In \emph{A Book at Bed Time}, Episode 38 (Season 3, Episode 12), 0:00--2:17.
  % \item \enquote{Peasant Scene}.\\
  %   In \emph{Monty Python and the Holy Grail},  8:49--11:59.
\end{itemize}

However, it quickly became clear that an isolated study of the sketches would be far too limited to grasp the full extent of the subversion that took place within each episode.  Furthermore, the diachronic study of the episodes was invalidated by the lack of adequation between the times at which the episodes were written and shot, and the order in which they were broadcast.  In face of those setbacks, a presentation of the framework within which to situate the politicity of the \emph{Flying Circus} was deemed enough, especially since it delineated paths for further research, many of which were mentioned in footnotes.

The crux of the research lies in the diachrony of comedy and satire from their literary roots to their modern reinterpretations.  The philosophical significance that is explored in \textcite{mcgowan2017} ought to be explored on a much wider scale, and so should the political significance that was developed in the current article.  If comedy truly is a reflection on subjectivity, it could be a prime candidate for a Foucauldian geneaology informed by \textcite{foucault1975, foucault1972}, more specifically the regimes of truth that they develop.  Just as any work in literature can be thought of as being a snapshot of their \emph{Zeitgeistser}, comedy could be seen as a truer account of those contexts because of the greater amount of immediacy that is permitted by levity, and the politicity that is conveyed under its guise.

One case-study that is all but designated in the current article is the political climate of Britain during the 1970s, and its superimposition on the work of Monty Python or other popular comedies of the time.  As a transitional decade between the counter-cultures of the 1960s and the ushering of the neoliberal age in the 1980s, it remains widely understudied, \textcite{forster2010} going so far as to refer to it as \enquote{[t]he Lost Decade} with a general \enquote{\emph{disillusionment} with democracy, government and other agencies [emphasis added]}.  Between partisan dealignment and the gradual switch to a radically different political paradigm, the genesis of the \enquote{depoliticisation of the masses} might be traced back to that period.  Since Monty Python had an enormous influence on the modern satiric and pseudo-satiric shows, the study of its perceived relevance in a political context might help situate the commercial co-optation of the term \enquote{satire}.

The study of humour and comedy has proven to be a wildly intriguing research topic.  From its philosophical conceptualisation by \textcite{bergson1900}, and after a tryst with almost every field in the Humanities, it is interesting to see it return to its philosophical roots.  The author hopes that this preliminary study will have proven to be enough to pique the reader’s interest, but also to vindicate the serious study of silly topics in their eyes.

{\setstretch{1.2}
  \printbibliography%
}

\end{document}